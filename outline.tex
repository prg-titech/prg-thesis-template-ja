\documentclass[11pt, a4paper, oneside, twocolumn]{jsarticle}
\usepackage{url}
\usepackage[dvipdfmx]{graphicx}
\usepackage{geometry}
\usepackage{comment}
\geometry{top=1cm, bottom=2cm, left=1.5cm, right=1.5cm}
\pagestyle{empty}

\title{%
(タイトル, 適宜改行すること)
}

% \author{東京工業大学 情報理工学院 数理・計算科学系\\xxBxxxxx 著者名\\指導教員 増原英彦教授}
\author{%
東京工業大学大学院 情報理工学研究科 数理・計算科学系\\
(学籍番号) (著者名)\\
指導教員 増原英彦 教授}

% 提出日?発表日?
% 無くても良い。空欄にすれば省略される。
\date{}

\begin{document}
\maketitle

\section{イントロ(本研究の動機)}
読者は以下のような内容を知りたい。
\begin{itemize}
\item \textbf{どの分野}の研究ですか?
\item \textbf{どのような課題}が研究を動機付けていますか?
\item 研究は\textbf{その課題にどのように関連}していますか?
\item \textbf{研究の目的}はなんですか?
\end{itemize}
通常の研究論文では、その研究の新規性(なぜ画期的なのか、技術的に重要なのか)を主張する必要があるが、卒論では義務ではない...と思う。

\section{本研究の背景知識 or 課題の詳細化}
読者は以下のような内容を知りたい。
\begin{itemize}
\item あなたの研究を理解するために必要な概念はなんですか?
\item 提示された課題はどこが問題なのか?
\end{itemize}

\section{本研究の内容}
研究の目的の達成のために、あなたは何をしましたか?
\begin{itemize}
\item 目的の達成のための方針やアイデアを提示する。
\item そのアイデアは何故目的に貢献しているのかを説明する。
\end{itemize}
これらより先に技術の詳細を書くべきでない。実装や言語定義など。

\section{本研究の評価}
% 大体以下のようなもの
% - ユーザー実験をした
% - 何らかの性質を満たしていることを証明した
% - 問題を反映したケーススタディをした
% - 性能測定をした
\begin{itemize}
\item やったことが研究の目的に合致していることは何によって担保されていますか?
\item まだそこまで到達していないなら、何をすることで研究の目的が達成されることを評価する計画ですか?
\end{itemize}
大体以下のようなプロセスであることが多い。
\begin{itemize}
\item Xを確かめるためにユーザー実験をした。その結果は〜だった。
\item Xを満たす言語であることを証明した。定理の主張は〜である。証明方法は〜を使った。
\item Xを確かめるためにケーススタディを行った。ケーススタディは動機となった課題を反映して〜のように設計した。その結果は〜だった。
\item Xを確かめるために性能測定をした。ベンチマークは目的を反映して〜に設計した。その結果はどうだった。
\end{itemize}

\section{結論 and/or Future work}
\begin{itemize}
\item この研究は何をしましたか?
\item この研究を行ったことでどのような研究が新たに可能になりますか?
\end{itemize}
卒論本体では、結論の前に関連研究も書きましょう。

{\scriptsize
% \renewcommand{\refname}{} % 「参考文献」の見出しの文字を空白にする
\begin{thebibliography}{1} 
 \bibitem{blockly} Blockly. https://developers.google.com/blockly.
\end{thebibliography}
}

\end{document}